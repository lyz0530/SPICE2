\chapter{SPICE输入语法}
\label{App:3}

SPICE程序的输入文件\footnote{在本附录中,短语"file"(文件)和"input deck"(输入卡)混合使用,短语"card"(卡)和“line”(线)也一样。}定义了电路,指明了要执行什么样的分析,以及产生什么样的输出。输入格式是一种“自由格式”类型,不要求在输入行的特定列位置输入数据。如果可能,程序会为没有指定值的参数提供合理的“默认”值。因此,用户只需要用一种简单的格式指定最少量的信息来仿真电路。

数据域由一个或者多个空格,一个逗号,一个等号,一个右括号,或者一个左括号来分界。分界符前面的或者后面的空格被忽略。一行输入通过在续行的第一域之前加入一个加号(+)就可以和下一行连续起来。但是注意,一个特定的数据域不能在行之间被分开,因为行尾被当作了分界符。

名称域以一个英文字母开头。只有名称的前七个字母数字字符会被程序保留。因此,名称“VERYLONGNAME”被缩写成名称“VERYLON”。名称用来识别电路元件,模型参数,输出变量,和程序关键字。

\begin{figure}[htbp]
\small
    \centering
    \includegraphics[width=0.7\textwidth]{figure/Appendix-A3/表A3.1.png}
    \caption{SPICE缩放因子}
    \label{表A3.1}
\end{figure}

数字域可以是一个整数,比如12或者-44,一个浮点数,比如3.14159或者-6.32,或者用指数表示的数字,比如1E-14或者-2.73E3。任何数字后面都可以跟着表\ref{表A3.1}中给出的缩放因子中的一个。注意数字和它的缩放因子之间不可以有空格;域2PF和两个域2\quad PF不是等价的。紧随数字域的非缩放因子的字母被忽略。紧随缩放因子的字母也被忽略。因此,10,10V,10VOLTS,和10HZ全部表示数字10,而M, MA, MSEC,和MMHOS全部表示缩放因子$10^{-3}$。数字域1000,1000.0,1000HZ,1E3,1.0E3,1KHZ,和1K表示数字$10^{3}$是等价的。

空白域特别地由两个逗号指定,逗号可以选择用空格分开。空白域等价于0数值域。因此,卡
.MODEL QMOD NPN ,,\quad,\quad,4.7,,,\quad,19.0等价于
.MODEL QMOD NPN 0\quad0\quad0\quad4.7\quad0\quad0\quad0\quad19.0

输入的第一行是标题;标题的内容逐字打印在SPICE输出的每一节的头部。deck中的最后一张card是一张.END card。.END card唯一的目的是表示SPICE输入结束。除了标题card和.END card,输入卡片的顺序随意。

通过把每个仿真的SPICE deck包含进输入,在一次SPICE运行中可以仿真几个电路。程序读输入直到遇到.END card,然后执行合适的仿真,随后再次读输入直到遇到下一张.END card。该过程一直进行直到所有的输入卡被读完。但是注意,如果一张.END卡被漏掉,那么两份分开的输入deck会被读作单一的一份输入deck,所以没有一个仿真会运行成功。

\section{SPICE电路描述}
电路通过一套元件卡来定义。电路中的每个元件被分配了一个独有的字母数字名称。电路中的每个节点被分配了一个独有的整数。节点数字0保留给了地节点。

为了确保对电路的数学描述唯一,在电路互联上得施加一定的限制。特别地,电路中的每个节点,包括地节点,必须至少有两个元件与它连接。电路中的每个节点必须有一条接地的DC通路。电路不能含有电压源和/或电感的环路。

电路中的每个元件由含有元件名称,连接该元件的电路节点,以及决定了该元件电气特性的参数的值的元件卡指定。元件名称的第一个字母指定了元件类型。

十二个SPICE元件类型的格式在表\ref{表A3.2}中给出。括号中的数字域是可选的。名称AC和OFF是程序的关键字,这些会在本附录后面的章节解释。表\ref{表A3.2}中的字符串“XXXXXX”,“YYYYYY”,和“ZZZZZZ”表示任意的字母数字字符串。电阻名称,例如,必须以字母R开头,而且能够含有一到七个字符。因此,R,R1,RSE,ROUT,和R3AC2ZY都是有效的电路名称。

\begin{figure}[htbp]
\small
    \centering
    \includegraphics[width=0.7\textwidth]{figure/Appendix-A3/表A3.2.png}
    \caption{SPICE元件卡格式}
    \label{表A3.2}
\end{figure}

\section{电阻}
电阻名称从字母R开始。电阻卡含有名称,正节点,负节点,以及电阻值(单位ohms):

RXXXXXX N+ N- value

电阻值不能是0或者负的。电阻两个节点的先后顺序不重要。

例如:

R37 19 14 1K    

RFEED 37 3 1MEG

\section{电容}
电容的名称以字母C开头。电容卡含有名称,正节点,负节点,电容值(单位farads),以及可选的初始条件:

CXXXXXX N+ N- value [incond]

电容值不能是0或者负的。初始条件是从正节点,流过电容,到负节点的0时刻的电容电流(单位amps)。如果没有指定初始条件,提供的是0值。

例如:

CBYP 13 0 1UF

COSC 17 19 1OUF 7.6MA

\section{电感}
电感名称用字母L开头。电感卡含有名称,正节点,负节点,电感值(单位henries),以及可选的初始条件:

LXXXXXX N+ N- value [incond]

电感的初始条件是0时刻的电感电压(单位volts)。如果没有指定初始条件,提供的是0值。

例如:

LSHUNT 13 7 1MH

L31 2 3 1OUH 4.7VOLTS

\section{耦合电感}
电感之间的耦合用耦合电感卡指定。耦合元件名称用字母K打头。该卡含有名称,两个耦合电感的名称,以及耦合系数的值:

KXXXXXX LYYYYYY LZZZZZZ value

耦合系数的绝对值必须小于1。给耦合系数指定一个负值等价于反转耦合电感中任意一个的极性。两个耦合电感,LYYYYYY和LZZZZZZ,可以用任意一种顺序被输入。

例如:

K43 LAAA LBB 0.999

\section{独立电压和电流源}
独立电压源名称以字母V开头,而独立电流源名称以字母I开头。这些元件卡含有源名称,正节点,负节点,DC和瞬态值,而字符AC之后会跟着该ac值的幅度和相对相位:

VXXXXXX N+ N- [dctrval] [AC acmag [acphs]]

IXXXXXX N+ N- [dctrval] [AC acmag [acphs]]

无论是dc/瞬态值还是ac值,又或者两者都会被忽略;如果没有指定值,提供的是0值。

例如:

VCC 17 0 5.0

IIN 1 3 AC 1

IIN 1 7 16 0.168 AC 0.3 45.0

VMEAS 16 17

对于瞬态分析,独立源可以被分配一个时间相关的值。如果给一个源分配了一个时间相关的值,那么0时刻的值会被用在DC分析上。有三种可行的时间相关源函数:PULSE,SIN,和EXP。

\begin{figure}[htbp]
\small
    \centering
    \includegraphics[width=0.7\textwidth]{figure/Appendix-A3/图A3.1.png}
    \caption{PULSE函数。}
    \label{图A3.1}
\end{figure}

PULSE函数如图\ref{图A3.1}中所示。该函数含有七个参数:初始值,脉冲值,延迟值,上升时间,下落时间,脉冲宽度,和周期:

PULSE (V1 V2 TD TR TF PW PER)

只有初始和脉冲值必须被指定;对于剩余参数,如果它们没有被指定,那么会提供图\ref{图A3.1}中给出的默认值。变量TSTEP和TSTOP是打印增量,停止时间在.TRAN卡上指定,关于这点会在本附录的最后一节解释。

\begin{figure}[htbp]
\small
    \centering
    \includegraphics[width=0.7\textwidth]{figure/Appendix-A3/图A3.2.png}
    \caption{SIN函数。}
    \label{图A3.2}
\end{figure}

SIN函数如图\ref{图A3.2}中所示。该函数含有五个参数:偏置值,幅度,频率(单位Hz),延迟时间,以及衰减因子:

SIN(V0 VA FREQ TD THETA)

SIN函数由下面的方程定义
\begin{equation}
    V=\left\{\begin{matrix}
V_0 & t\leq t_d  \\
V_0 + V_A\{exp[-\theta(t-t_d)]\}\{sin[2\pi f(t-t_d)]\} & t \geq t_d \\
\end{matrix}\right.
\label{eq:A3.1}
\end{equation}
参数VO, VA,和FREQ必须被指定。剩余参数的默认值,TD和THETA,在图\ref{图A3.2}中给出。

\begin{figure}[htbp]
\small
    \centering
    \includegraphics[width=0.7\textwidth]{figure/Appendix-A3/图A3.3.png}
    \caption{EXP函数。}
    \label{图A3.3}
\end{figure}

EXP函数在图\ref{图A3.3}中给出。该函数含有六个参数:初始值,终值,领先边延迟,领先边时间常数,下落边延迟,以及下落边时间常数:

EXP (V1 V2 TD1 TAU1 TD2 TAU2)

EXP函数由下面的方程定义
\begin{equation}
    V=
\left\{\begin{matrix}
V_1 & 0 \leq t \leq t_{d1}  \\
V_1 + (V_2 - V_1)[1-exp(-\frac{t-t_{d1}}{\tau_1})] &t_{d1} \leq t \leq t_{d2}  \\
V_1 + (V_2 - V_1)[1-exp(-\frac{t-t_{d1}}{\tau_1})] + (V_1 - V_2)[1-exp(-\frac{t-t_{d2}}{\tau_2})] & t \geq t_{d2} 
\end{matrix}\right.
\label{eq:A3.2}
\end{equation}
初始值和脉冲值必须被指定。剩余值的默认值在图\ref{图A3.3}中给出。

\section{线性压控电流源}
线性压控电流源的名称以字母G打头。这个元件卡含有名称,正节点,负节点,正控节点,负控节点,跨导值(单位mhos),和延迟(单位secs)的可选值:
GXXXXXX N+ N- NC+ NC- value [delay]

如果没有指定延迟,那么会提供一个零值。

例子:

G35 2 0 5 0 0.1MMHO

GXZ 17 3 19 7 20MMHO 5NS

\section{非线性压控电流源}
非线性压控电流源的名称以字母N打头。元件卡包含名称,正节点,负节点,正控节点,负控节点,初始条件,和多项式系数:

NXXXXXX N+ N- NC+ NC- incond P1[P2 P3 \dots P20]

初始条件是控制节点之间电压的初始“猜测”,必须被指定。程序使用该初始条件获得电路的DC工作点。取得收敛之后,程序继续迭代以获取控制电压的准确值。初始条件不需要准确,但是为了减少DC工作点的计算代价,初始条件应该相当接近真实的控制电压。

只有第一个多项式系数需要被指定;剩余的多项式系数没有被指定的话会被设置为0。指定的系数总数不能超过20个:

例子:

N31 21 0 21 0 0.0 0.2 0.05 0.007

\section{半导体器件}
SPICE程序中包含的四种半导体器件模型含有很多参数。而且,电路中很多器件经常由同一套器件模型参数定义。出于这些原因,一套器件模型参数被定义在一张分立的.MODEL卡上,并被分配了一个特有的模型名称。SPICE中的器件元件卡之后参考该模型名称。这种方案减轻了在每个器件元件卡上重复指定模型参数的需求。

每张器件元件卡含有器件名称,连接该器件的节点,以及器件模型名称。另外,每个器件可能会被指定两个可选参数:一个面积因子和一个初始条件。

面积因子决定了指定模型的等效并行器件的数目。例如,面积因子为3.0预示着该器件等价于三个指定模型名称的器件。

初始条件被包含进来是为了提高含有超过一个稳态的电路的DC收敛性。如果一个器件被指定为OFF,那么DC工作点会在该器件端口电压设置为0的情况下被决定。获得收敛后,程序会继续迭代以获得端口电压的准确值。如果电路的DC稳态超过一个,那么OFF选项会被用来强制解与期望的状态一致。如果在实际中一个器件导通但该器件被指定为了OFF,那么程序会获得正确的解(假定求解收敛)但需要更多次的迭代,因为程序必须独立地收敛到两个不同的解。

\section{二极管}
二极管名称以字母D开头。一张二极管卡含有名称,阳极节点,阴极节点,模型名称,面积因子的可选值,以及可选的初始条件:

DXXXXXX NA NC MODEL [area] ["OFF"]

例子:

DBRI 23 25 DM1

DS1 24 27 SBD37X 2.5 OFF

\section{双极型结晶体管}
BJT名称以字母Q开头。一张BJT卡含有名称,集电极节点,基极节点,发射极节点,模型名称,面积因子的可选值,以及可选的初始条件:

QXXXXXX NC NB NE MODEL [area] ["OFF"]

例子:

Q23 4 6 9 QMODEL

QLS 23 26 29 FAST 2.0 OFF

\section{结场效应晶体管}
JFET名字以字母J开头。JFET卡含有名称,漏节点,栅节点,源节点,模型名称,面积因子的可选值,和可选的初始条件:

JXXXXXX ND NG NS MODEL [area] ["OFF"]

例子:

J34 6 7 8 JMOD2

JIN 1 17 20 JM0 1.5 OFF

\section{MOSFET的}
MOSFET名称以字母M开头。MOSFET卡含有名称,漏节点,栅节点,源节点,衬底节点,模型名称,面积因子的可选值,和可选的初始条件:

MXXXXXX ND NG NS NB MODEL [area] ["OFF"]

例子:

MLONG 23 25 27 28 MOS1 0.1

MIN 12 14 16 1 MOS2

MGATE 11 14 17 10 MOS2 2.5 OFF

\section{.MODEL卡}
给定模型名称的器件模型参数被指定在一张.MODEL卡上。该卡含有单词“.MODEL”,模型名称,器件模型类型,以及模型参数值:

.MODEL name type [parameter values]
该模型名称可以是一个随意的名称。模型类型是表\ref{表A3.3}中给出的七个关键词中的一个。注意器件极性被指定为模型定义的一部分。

\begin{figure}[htbp]
\small
    \centering
    \includegraphics[width=0.7\textwidth]{figure/Appendix-A3/表A3.3.png}
    \caption{模型类型关键词。}
    \label{表A3.3}
\end{figure}

每个SPICE模型参数通过唯一的一个关键词引用。没有指定的模型参数被分配了默认值。二极管模型参数,关键词,和默认值在表\ref{表A3.4}中列出。表\ref{表A3.5}含有BJT模型的参数关键字和默认值。JFET参数关键字和默认值在表\ref{表A3.6}中给出,而表\ref{表A3.7}中含有MOSFET模型参数的关键词和默认值。这四种模型在本论文的附录\ref{App:2}中进行了详细描述。

\begin{figure}[htbp]
\small
    \centering
    \includegraphics[width=0.7\textwidth]{figure/Appendix-A3/表A3.4.png}
    \caption{SPICE二极管模型参数。}
    \label{表A3.4}
\end{figure}

\begin{figure}[htbp]
\small
    \centering
    \includegraphics[width=0.7\textwidth]{figure/Appendix-A3/表A3.5.png}
    \caption{SPICE BJT 模型参数。}
    \label{表A3.5}
\end{figure}

\begin{figure}[htbp]
\small
    \centering
    \includegraphics[width=0.7\textwidth]{figure/Appendix-A3/表A3.6.png}
    \caption{SPICE JFET 模型参数。}
    \label{表A3.6}
\end{figure}

\begin{figure}[htbp]
\small
    \centering
    \includegraphics[width=0.7\textwidth]{figure/Appendix-A3/表A3.7.png}
    \caption{SPICE MOSFET 模型参数。}
    \label{表A3.7}
\end{figure}

模型参数可能会被用两种不同的方法指定。第一种,也是最简单的一种,方法是指定参数关键词,之后是等号(=),之后是参数值。例如,下面的卡片定义了模型名称QX:

.MODEL QX\footnote{原文中为QN} NPN (BF=30, RB=50, IS=1E-15, TF=INS)
模型QX是一个npn BJT,$B_F$等于30,$r_b$等于50,$I_S$等于$10^{-15}$,而$t_F$等于$10^{-9}$。没有指定的参数被分配的默认值在表\ref{表A3.5}中给出。

第二种定义模型参数值的方法是通过一串数字字符串。这些值的顺序必须与表\ref{表A3.4}-\ref{表A3.7}中给出的关键字的顺序一致。参数位置用连续的逗号跳过。下面的模型卡

.MODEL QX\footnote{原文中为QN} NPN 30,,1E-15,50,,,,,,,,,,1NS
等价于之前定义的模型QX。

\section{子电路}
包含SPICE元件的子电路可以用与器件模型相似的方式来定义和引用。输入deck中的子电路是由一组元件卡定义的;无论子电路何时被引用,程序随后会自动插入元件组。对子电路的规模或者复杂度没有限制。图\ref{图A3.4}中展示的三种电路构成了可能的子电路。

\begin{figure}[htbp]
\small
    \centering
    \includegraphics[width=0.7\textwidth]{figure/Appendix-A3/图A3.4.png}
    \caption{SPICE可能的子电路。}
    \label{图A3.4}
\end{figure}

子电路名称以字母X开头。器件卡包含名称,子电路节点,和子电路名称。图\ref{图A3.4}中所示的多发射极器件可以通过下面的卡在输入deck中指定:

X13 2 4 17 3 1 MULTI

节点的顺序必须与它们在子电路卡中定义的一致。

\section{.SUBCKT卡}
子电路由.SUBCKT卡定义。该卡含有字母“.SUBCKT”,子电路名称,和外部节点。一个子电路可能有1到20个外部节点。.SUBCKT卡之后的元件卡组定义子电路。子电路定义中的最后一张卡在.FINIS卡中\footnote{原文缺字,此句是推断翻译的}。图\ref{图A3.4}中展示的多发射极器件由下面的卡组指定:

.SUBCKT MULTI 1 2 3 4 5\\
Q1 1 2 3 QLONG\\
Q2 1 2 4 QLONG\\
Q3 1 2 5 QLONG\\
.FINIS

发生在.SUBCKT卡和.FINIS卡之间的模型或者控制卡与它们在级联之外有着相同的影响。子电路不能级联;也就是说,子电路中不能包含其他子电路。

\section{控制卡}
除了指定电路,SPICE输入deck还指定要执行的分析和用户希望的输出。这些额外的指定都是通过表\ref{表A3.8}中总结的SPICE控制卡来输入的。这些控制卡的语法在本附录下面的章节会被描述。

\begin{figure}[htbp]
\small
    \centering
    \includegraphics[width=0.7\textwidth]{figure/Appendix-A3/表A3.8.png}
    \caption{SPICE 控制卡。}
    \label{表A3.8}
\end{figure}

\section{标题卡}
该卡必须是deck中的第一张卡。它含有任务的字母数字标题,会给程序的各种输出章节打印标题。

\section{.END卡}
该卡含有字符“.END”,必须是deck中的最后一张卡。如果只运行一次仿真,.END卡会被忽略。

\section{评论卡}
该卡容许对输入deck进行记录。该卡第一个非空字符是一个星号(*);该卡虽然用列表打印,但是会被程序忽略。

\section{.OUTPUT卡}
该卡定义了输入变量,并指定了要打印或者绘制的输出的分析。输出也许会是节点电压差,或者电压源电流。另外,在AC分析中,输出噪声,输入噪声,以及任一失真测量都可以被指定为输出以及打印或者绘制。

.OUTPUT卡的一般语法如下所示:
\begin{equation*}
.OUT \begin{bmatrix}
VXXXXXX \quad N+ \quad N- \\
IXXXXXX \quad VYYYYYY \\
ONOISE \\
RINOISE \\
HD2 \\
HD3 \\
SIM2 \\
DIM2 \\
DIM3
\end{bmatrix}\quad "PRINT"\quad [options]\quad "PLOT"\quad [options]
\end{equation*}
对于电压输出,输出名称之后是正节点和负节点。对于电流输出,输出名称之后是电压源名称。如果电流从源正节点,经过源,流到源负节点,那么电流是正的。

关键词ONOISE和RINOISE分别代表着输出噪声和反射输入噪声。关键词HD2,HD3,SIM2,DIM2,DIM3代表着失真分析中的失真产品。这些产品会在本附录后面的章节详细介绍。

\begin{figure}[htbp]
\small
    \centering
    \includegraphics[width=0.7\textwidth]{figure/Appendix-A3/表A3.9.png}
    \caption{SPICE 输出选项。}
    \label{表A3.9}
\end{figure}

在用SPICE执行的任意一种分析中,输出变量可以被打印出或者绘制出。.OUTPUT卡也控制着输出变量何时以及如何被作为输出。SPICE中可行的各种输出选项在表\ref{表A3.9}中列出。通过在关键词PRINT后添加期望的输出选项,输出可以用表格的形式打印出来。例如,下面的.OUTPUT卡

.OUTPUT V7 7 0 PRINT DC TR MAG PHS

会打印出DC转移曲线分析和瞬态分析中的节点电压7。另外,AC分析中V7的幅度和相位都会被打印出来。

在关键词PLOT之后添加期望的输出选项可以获得一个输出的线打印图。在每个输出选项后面也许会加入PLOT限制。如果没有包含plot限制,程序将自动确定plot限制。下面的.OUTPUT卡

.OUTPUT v7 7 0 PLOT DC TR 0 5 MA

会线打印绘制出DC转移曲线分析,瞬态分析,和AC分析中产生的V7。在AC分析中,V7的幅度会被绘制。程序在瞬态绘制中使用plot限制(0,5),而在DC和AC分析绘制中会确定合适的限制。

\section{.DC卡}
该卡定义了DC转移曲线源和限制。它含有字母“.DC”,之后是电压或者电流源名称,之后是起始值,停止值,以及加的增量。例如,下面的卡\\
.DC VIN 0.5 5.0 0.25\\
会产生电压源VIN,以0.25volts为增量从0.5volts扫描到5volts。

\section{.TF卡}
该卡定义了DC小信号分析的小信号输入和输出。它含有字母“.TF”,之后是小信号输出变量和小信号输入源。如果下面的卡\\
.TF VOUT VIN\\
被包括进输入deck,那么作为DC工作点分析的一部分,程序会计算转移函数(VOUT对VIN的比率),VIN处的小信号输入阻抗,以及VOUT处的小信号输出阻抗。VOUT必须是.OUTPUT卡中已经定义的输出变量,而VIN必须是已经在电路描述中定义的源。

\section{.SENS卡}
该卡含有字母“.SENS”,之后是一个到十个已经在.OUTPUT卡中定义的输出变量名称。作为一个例子,如果下面的卡\\
.SENS VOUT V12 VINT\\
被包含在输入deck中,那么程序会确定VOUT,V12和VINT关于每个电路参数的DC小信号敏感度。JFET和MOSFET模型参数没有被包含在敏感度计算中。

\section{.TRAN卡}
该卡含有瞬态分析仿真控制。该卡最少的控制信息需要包含字母“.TRAN”,之后是打印时间增量和停止时间。程序会从0时刻开始,用给定的打印增量计算每个时间点直到停止时间。

开始时间可以被贴在停止时间之后。在这种情况下,程序会从0时刻开始,计算每个时间点直到停止时间;但是,只有开始时间和停止时间之间的时间点会被打印。这个选项是为那些电路被允许到达稳态解的仿真设计的。

\section{.FOUR卡}
Fourier分析可以在.FOUR卡上指定。该卡含有字母“.FOUR”,之后是输出变量名称和基本频率。如果被指定,程序会计算指定输出变量响应的前九个Fourier谐波。

\section{.AC卡}
该卡含有AC分析控制。.AC卡含有字母“.AC”,之后是下面三种频率变量类型之一:
\begin{enumerate}
    \item 字母“DEC”之后是每十倍频的点,起始频率,和停止频率。
    \item 字母“OCT”之后是每倍频的点,起始频率,和停止频率。
    \item 字母“LIN”之后是点数,起始频率,和停止频率。
\end{enumerate}
如果.AC卡被包含进deck中,那么程序会在.AC卡上指定的频率范围内执行电路的AC分析。

\section{.NOISE卡}
该卡控制着电路的噪声分析。它含有字母“.NOISE”,之后是噪声输出参考,噪声输入参考,以及总结间隔。程序会在指定的输出处计算等效输出噪声,同样也会在指定的输入处计算等效的输入噪声。另外,电路中的每个噪声生成器的贡献会在每n个点上被打印出来,这里n是总结间隔。如果总结间隔是0,那么就不会发生总结打印输出。

\section{.DISTO卡}
该卡含有控制失真分析的参数。失真分析,反过来,会与AC小信号分析一起被执行。这种子分析会决定谐波失真产品和互调失真产品。用户指定输出负载电阻,RLOAD。每个失真产品是在特定失真频率处传到RLOAD的功率相对在基本频率处传到RLOAD的功率。

谐波失真产品是输出中的较高谐波的相对幅度。如果电路在频率$f_1$处含有一个正弦激励,那么HD2,二阶谐波失真产品,就是在频率2$f_1$处RLOAD中的相对功率。类似地,HD3,三阶谐波失真产品,是在频率$3f_1$处RLOAD中的相对功率。

互调失真产品是在频率$f_1$和$f_2$处用两个正弦激励测得的。SIM2,二阶和互调失真产品,是在频率$f_1+f_2$处传递的相对功率。类似地,DIM2\footnote{原文中为DIM3},二阶差互调失真产品,是在频率$f_1-f_2$处传递的相对功率。最后,DIM3,三阶差互调失真产品,是在频率$2f_1-f_2$处传递的相对功率。

这些失真测量中的任何一种都可以在.OUTPUT卡上的每个频率点上被打印或者绘制出来。另外,电路中每个非线性对整体失真的贡献也可以被打印出来。

.DISTO卡含有字母“.DISTO”,之后是负载电阻名称,总结间隔,参考功率水平,两种失真频率的比率,以及$f_2$信号的幅度,如下:\\
.DISTO RLOAD INTER SKW2 REFPWR SPW2\\
RLOAD是输出负载电阻的名称,在那里所有的功率产品都会被计算。程序会在每n个点处打印每个失真源的贡献,这里n是总结间隔。如果总结间隔是0,那么就不会发生总结打印输出。SKW2是$f_2/f_1$的比率;$f_1$,当然,是.AC卡控制的基本频率。如果SKW2没有被指定,那么提供的值是0.9。REFPWR是参考功率水平,被用在计算失真产品中。如果没有被指定,使用的值是1mW;也就是说,所有的失真产品被用dBm表示。SPW2是$f_2$信号的幅度相对基本$(f_1)$信号的幅度。如果SPW2没有被指定,那么假定值为1。

\section{.TEMP卡}
该卡含有仿真执行时的温度。模型数据通常指定在27$^{\circ}$C (300 K)。如果该卡被删除了,仿真也会在27$^{\circ}$C被执行。温度卡含有字母“.TEMP”,之后是一到五个温度。例如,下面的卡\\
.TEMP -25 0 25 50 100\\
会使得SPICE仿真在-25$^{\circ}$C,0$^{\circ}$C,25$^{\circ}$C,50$^{\circ}$C和100$^{\circ}$C进行。

\section{.OPTIONS卡}
该卡容许用户为了特定的仿真目的重置程序控制。该卡含有字母“.OPTIONS”,之后是下面这些选项中的任意一个,顺序无所谓:\\
ACCT\quad 把程序的各种章节的执行时间打印出来,同时也打印出其他的记录信息。\\
LIST\quad 把输入数据的总结列表打印出来。\\
NOMOD\quad 压缩模型参数的打印输出。\\
NODE\quad 把节点列表打印出来。\\
OPTIONS\quad 把选项的预设值打印出来。\\
GMIN=X\quad 重置GMIN的值,程序容许的最小的电导。GMIN的默认值是$10^{-12}$。\\
RELTOL=x\quad 重置程序的相对误差容限。RELTOL的默认值是0.001(0.1\%)。
ABSTOL=x\quad 重置程序的绝对误差容限。ABSTOL的默认值是$10^{-12}$。\\
TRTOL=x\quad 重置瞬态误差容限。TRTOL的默认值是10.0。\\
CHGTOL=x\quad 重置程序的电荷容限。CHGTOL的默认值是$10^{-14}$。\\
TNOM=x\quad 重置正常温度。TNOM的默认值是27$^{\circ}$C (300 K)。\\
ITL1=x\quad 重置DC迭代限制。默认值是100。\\
ITL2=x\quad 重置DC转移曲线迭代限制。默认值是20。\\
ITL3=x\quad 重置低瞬态分析迭代限制。默认值是4。\\
ITL4=x\quad 重置瞬态分析时间点迭代限制。默认值是10。\\
ITL5=x\quad 重置瞬态分析总的迭代限制。默认值是5000。\\
LIMTIM=x\quad 重置程序的时间限制。默认值是2000ms。\\
LIMPTS=x\quad 重置在一次DC,AC,或者瞬态分析中计算的总点数。默认值是201。\\
LVLCOD=x\quad 如果x是2,矩阵解的机器码会被产生出来。否则,不会产生机器码。LVLCOD的默认值是1。\\
LVLTIM=x\quad 如果x是1,使用迭代时间步长控制。如果x是2,使用截断误差时间步长。LVLTIM的默认值是2。