\chapter{前言}
\label{chap:1}
电子电路设计需要精确的方法来评估电路性能。对于离散电路设计,传统的“面包板”是一种测量电路电子特性的便利方法。为了隔绝边际或者不可接收的设计的起因,电路可以按照意愿被修改和监测,而设计提高能被立刻做出。因为面包板近似模仿了最终会构造的电路,所以实验室的测量设备就会反映出最终电路性能的精确特征。

集成电路(IC)的设计提出了一个完全不同的问题。面包板和生产出的IC明显没有相似性。面包板中的寄生参数完全不同于集成电路中的寄生元件。因为这个原因,面包板测量经常会对电路性能产生出不精确的特性。IC的制造和测试当然会验证一个可接受的设计;然而,IC非常小的尺寸排除了那些用来改良边缘或者不可接受设计的大量的监测和修正手段。

仿真电子电路电气性能的计算机程序可以规避电路表征中遇到的很多实际问题。电路用数学项和与执行的典型实验室测量办法相对应的数值分析步骤来代表。电路设计人员选择进行的分析类型以及,打个比方,施加到电路上的测量手段。仿真程序的输出因此仿真了实验室测量的结果。而且,电路仿真可以提供关于电路性能的信息,而这本质上是无法从实验室测量处获得的。

SPICE\cite{ref-1}是目前可行的几个成功的电路仿真程序之一。SPICE程序主要由作者开发,是University of California,Berkeley的电子研究实验室的集成电路组开发的很多仿真程序中的一个。SPICE从CANCER程序进化而来\cite{ref-2,ref-3}。SPICE的第一个版本基本上在1972年完成。从那时起,该程序已有超过一百个复制品被分发给了各个大学和电子工业界的公司。

SPICE的广泛使用证明了该程序对大量电路仿真问题的实用性。这种使用也得到了关于SPICE程序中应用的计算方法的优点和缺点的宝贵经验基础。这种经验,反过来,促进了SPICE2的开发,也就是SPICE的第二个版本。

该论文描述了SPICE程序的开发和设计。提出了对电路仿真任务必不可少的计算方法,对SPICE程序中应用的特定算法进行了详细介绍。给出了这些算法对典型电子电路的性能,而SPICE中使用的技术也和其他可行的方法进行了对比。

\section{SPICE的描述}
SPICE是一个数字计算机程序,仿真电子电路的电气性能。这个程序会决定电路的静态工作点,电路的时域响应,或者电路的小信号频域响应。SPICE包含常见的电路部件,并且有能力仿真大多数的电子电路。SPICE程序含有大约10,000行FORTRAN IV和COMPASS声明,是为了适用于University of California,Berkeley的计算机中心的CDC 6400计算机\footnote{计算机中心的CDC 6400在由计算机中心工作人员开发的CALIDOSCOPE操作系统下工作。SPICE的FORTRAN部分是用RUNW.2编译器编译的,而汇编部分是用COMPASS 1.1汇编器汇编的。}。然而,程序也已经适用于IBM,Honeywell,UNIVAC,RCA,和PDP的计算机系统。

SPICE的输入语法属于free-format类型,不要求把数据输入到固定的列位置。程序对没有指明的电路参数提供合理的默认值,并且执行相当数量的错误检查以确保电路被正确地输入。初学者需要指明电路参数和仿真控制的最小个数,以此来获得合理的仿真结果。

\section{SPICE中的电路定义}
程序输入通过元件基础来定义在元件上要仿真的电路。目前在SPICE程序中包含的电路元件类型在表\ref{表1.1}中列出。这些元件的模型在本论文的附录\ref{App:2}中详细列出。SPICE元件包括电阻,电容,电感,耦合互感,独立电压和电流源,以及四种最常见的半导体器件:二极管,双极型晶体管(BJT),结型场效应晶体管(JFET),和绝缘栅场效应晶体管(IGFET或者MOSFET)。
\begin{figure}[htbp]
\small
    \centering
    \includegraphics[width=0.7\textwidth]{figure/Chapter1/表1.1.png}
    \caption{SPICE电路元件}
    \label{表1.1}
\end{figure}

举个例子最能展示SPICE程序的输入。图\ref{图1.1}是一个差分对放大器阶的原理图。电路中的每一个节点都被分配了一个独特的整数,数字0留给地节点。每一个元件电路都被分配了一个独特的名字。元件名字的第一个字母是识别元件类型的前缀。例如,电阻名字以字母R开头,BJT名字以字母Q开头。每一个SPICE元件的前缀字母都被包含进了表\ref{表1.1}。

\begin{figure}[htbp]
\small
    \centering
    \includegraphics[width=0.7\textwidth]{figure/Chapter1/图1.1.png}
    \caption{差分对电路的原理图}
    \label{图1.1}
\end{figure}

定义了差分对电路的SPICE输入Deck在图\ref{图1.2}中给出。该Deck中的第一张卡是标题卡;该卡的内容被打印成了SPICE输出的不同节的标题。该Deck中的最后一张卡是.END卡,只用来表示输入Deck的结束。除了标题卡和.END卡,这些卡的顺序对程序没有影响。

\begin{figure}[htbp]
\small
    \centering
    \includegraphics[width=0.7\textwidth]{figure/Chapter1/图1.2.png}
    \caption{差分对电路的SPICE输入Deck}
    \label{图1.2}
\end{figure}

电路中每一个元件用一张元件卡定义,该卡包含了元件名字,该元件连接的节点,以及定义了元件电气特性的参数的值。例如,电压源VCC把它的正节点连到了电路的节点8,负节点连到了地,而电压值是12 volt。电压源VIN的值专门用一个预定义的SIN函数指定;VIN是一个有着0 volt偏置,0.1 volt幅度(0.2 volt峰峰值),和5MHz频率的正弦波。

表\ref{表1.1}给出的线性元件只要求1个或者2个参数值就可以完全指定该元件的电气特性。然而,四个半导体器件的模型,包括BJT,是更复杂的,而且包含了很多参数。另外,电路中的几个器件通常可以用一套常见的模型参数描述。因此,在分离的.MODEL卡上可以方便地指定器件模型参数;这套参数被分配了一个独特的模型名字,之后可以被其他器件元件卡引用。在这个例子中,四个BJT模型共用同一套定义在.MODEL卡上的,被分配了QNL模型名字的模型参数。

\section{SPICE分析类型}
除了定义电路,程序输入指定要执行的分析类型和要产生的输出。该信息被输入进控制卡。电子电路的仿真通常要求三种基本分析的组合:DC分析,时域瞬态分析,和小信号AC分析。SPICE包含所有这三种分析能力。除此之外,几种子分析能力也被纳入进了SPICE程序。SPICE中可行的十种分析类型在表\ref{表1.2}中给出\footnote{小信号敏感分析,噪声分析,和失真分析在SPICE的第一个版本中进行了实现,但现在还没有被纳入SPICE2中。}。
\begin{figure}[htbp]
\small
    \centering
    \includegraphics[width=0.7\textwidth]{figure/Chapter1/表1.2.png}
    \caption{SPICE电路元件}
    \label{表1.2}
\end{figure}

\section{DC分析}
DC分析决定了电路的静态工作点。电路中的所有能量存储元件在DC分析中都会被忽略,其中会把电容看作开路,电感看作短路。在工作点分析的结论里,程序打印出电路节点电压,独立电压源电流,和电路总静态功耗。

对小信号工作来说,电路的扰动响应可以通过用等效的线性化模型对电路中的非线性元件建模决定。线性化模型的参数依赖,当然,静态工作点。这些线性化模型在本论文的附录\ref{App:2}给出。对于BJT的情况,使用了熟知的hybrid-pi模型\cite{ref-4}。所有非线性元件工作点,以及线性化的模型参数,都会在DC工作点分析的结论中被打印出来。

如果输出变量和输入电源被指定,SPICE也会决定DC,小信号转移函数值。该转移函数是输出变量对输入电源的小信号比值。另外,程序会决定电路的小信号输入电阻和小信号输出电阻。

SPICE也包含一个子程序,负责计算和打印指定输出变量相对每个电路参数的DC敏感度\footnote{关于JFET和MOSFET模型参数的敏感度目前在SPICE中没有被实现。}。有效临近网络概念\cite{ref-5,ref-6}被用来决定这些敏感度。

图\ref{图1.2}中展示的SPICE输入足够决定电路的DC工作点。SPICE程序的输出,关于图\ref{图1.2}中展示的输入,在图\ref{图1.3}到图\ref{图1.5}中给出。首先,程序读取输入并把输入内容列出,如图\ref{图1.3}所示。如果遇到错误,程序读完输入后终止。否则,电路总结,模型参数总结,以及节点列表会被打印,如图\ref{图1.4},而且电路数据结构会被构建并检查是否有错。最后,程序决定DC工作点和打印静态节点电压,电压源电流,非线性元件工作点,以及线性化的器件模型参数,如图\ref{图1.5}所示。

\begin{figure}[htbp]
\small
    \centering
    \includegraphics[width=0.7\textwidth]{figure/Chapter1/图1.3.png}
    \caption{SPICE的输入列表}
    \label{图1.3}
\end{figure}

\begin{figure}[htbp]
\small
    \centering
    \includegraphics[width=0.7\textwidth]{figure/Chapter1/图1.4.png}
    \caption{SPICE电路总结}
    \label{图1.4}
\end{figure}

\begin{figure}[htbp]
\small
    \centering
    \includegraphics[width=0.7\textwidth]{figure/Chapter1/图1.5.png}
    \caption{SPICE DC工作点的解}
    \label{图1.5}
\end{figure}

\section{DC转移特性}
针对指定输入源连续变化的输入值,DC工作点可以被重复决定,以此获得电路的一套DC转移特性。特定的输出变量(无论电压还是电流)的值针对每个输入源都进行了存储。在分析的结论里,这些输出变量可以用表格的形式列出或者用线打印机绘制。DC转移曲线被用来决定数字电路的噪声容限,以及模拟电路的大信号特性。比如,图\ref{图1.6}展示的就是SPICE线打印机绘制的差分对电路的节点5处的电压。为了产生这个图,输入源,VIN,以5mV的间隔从-0.25V变化到了0.25V。

\begin{figure}[htbp]
\small
    \centering
    \includegraphics[width=0.7\textwidth]{figure/Chapter1/图1.6.png}
    \caption{SPICE DC转移曲线的解}
    \label{图1.6}
\end{figure}

\section{非线性时域瞬态分析}
瞬态分析决定了电路对指定时间域输入的时间域响应。初始时间点,随意定义为时间点0,由之前的DC工作点的解决定。电路中的独立源的值是恒定的或者随时间变化的。用户指定的时间间隔(0,T)被分成离散的时间点,而程序从时间0开始在每一个连续的时间点上求解电路。电压或者电流输出变量在每一个时间点上都进行了存储,而且可以用表格的形式列出或者在分析的结论处绘制出来。为了确保得到准确的解,连续的时间点之间的间隔由程序控制\footnote{CANCER和SPICE的第一个版本没有时间步长控制。然而,SPICE2确实有了时间步长控制。}。输出数据是插值得到的,这样可以使得输出按照用户指定的打印间隔排布。

差分对电路瞬态分析的SPICE线打印机绘制的结果如图\ref{图1.7}所示。该图中给出的波形是电路中节点5处的电压;时间间隔是500ns,而打印增量是5ns。输入源,VIN,是偏置为0 volts,幅度为0.1 V (0.2 V 峰峰值),频率为5MHz的正弦波。

\begin{figure}[htbp]
\small
    \centering
    \includegraphics[width=0.7\textwidth]{figure/Chapter1/图1.7.png}
    \caption{差分对电路瞬态分析的SPICE线打印机绘制的结果}
    \label{图1.7}
\end{figure}

输出函数中包含了一个傅里叶分析子程序,可以决定指定输出的前9个傅里叶系数。该能力对评估近似正弦波形的傅里叶谐波失真成分是有用的。然而,拟合傅里叶级数到时域波形固有的不精确性会限制子分析能力在相对大的谐波失真值上的可用性。

\section{AC分析}
SPICE的小信号,频域分析部分对工作在小信号模式的模拟电路的设计是有用的。电路的扰动响应通过对非线性元件使用线性化模型能够得到。这些线性化模型的参数由DC工作点分析决定。小信号线性化等效电路在频域用相量法分析\cite{ref-7}。所有的电路电压和电流都是复变量,通常用幅度和相位的形式表示。独立源也有用幅度和相位项表示的复值。

AC分析通常是在一连串频率值上执行,以此来决定电路转移函数的频率响应。指定的电压和电流输出在每一个频率点上都有存储。在分析的结论里,这些输出可以被列出或者绘制为幅度,用dB表示的幅度,相位,实数部分,或者虚数部分。差分对电路中节点5处的电压的幅度和相位的绘制分别在图\ref{图1.8}和图\ref{图1.9}中给出。频率从1Hz变化到1GHz,绘制的精度是每十倍频十个点。输入源,VIN,被分配了单位1的幅度和0相位的值。

\begin{figure}[htbp]
\small
    \centering
    \includegraphics[width=0.7\textwidth]{figure/Chapter1/图1.8.png}
    \caption{SPICE在AC分析中幅度响应。}
    \label{图1.8}
\end{figure}

\begin{figure}[htbp]
\small
    \centering
    \includegraphics[width=0.7\textwidth]{figure/Chapter1/图1.9.png}
    \caption{SPICE在AC分析中相位响应。}
    \label{图1.9}
\end{figure}

\section{噪声分析}
SPICE的AC分析部分包含了增加的评估电子电路噪声特性的能力\cite{ref-3,ref-8,ref-9}。电路中每一个电阻都会产生热噪声,而电路中的每一个半导体器件除了器件欧姆电阻产生的热噪声外还会同时产生散粒噪声(shot noise)和闪烁噪声(flicker noise)。该噪声生成被建模为与每一个产生噪声的元件相关的等效独立源。SPICE中使用的噪声模型在本论文的附录\ref{App:2}给出。

因为电路中的等效噪声源是统计无关的,所以在指定输出中每一个噪声源对总体噪声的贡献必须分开计算。总输出噪声是单个噪声贡献的RMS和。噪声分析是用一种有效的方法执行的,该方法通过在每一个频率点上利用临近网络概念(adjoint network concept)\cite{ref-8,ref-10}来评估只有一个额外的AC解下的该RMS和。

噪声分析完成之后,整体的输出噪声和等效输入噪声都可以用表列出或者线打印机绘制。另外,电路中每一个噪声源的贡献可以在选择的频率点上打印出来。

\section{失真分析}
对相对高水平的失真来说,把周期波形的傅里叶级数拟合到时域响应的方法可以得到对波形失真部分的一个好的近似。低水平的失真通过频域失真分析\cite{ref-11,ref-12}可以更加准确地确定。SPICE中实现的失真分析目前能够评估二阶和三阶谐波失真,也可进行二阶和三阶的互调失真分析。

失真分析是通过用Volterra级数\cite{ref-13}近似每一个非线性元件模型的方程来执行的。每一个非线性元件随后可以用线性化的等效元件和一个独立源建模,该独立源代表该特定元件产生的失真。失真分析因此能类比噪声分析。对一个特定的失真组分,小信号电路包含几个“失真产生器”,而整体的失真在特定的输出被加了起来。然而,整体的失真组分是单个失真贡献的向量和,而不是其RMS和。

在失真分析的结论中,任一失真组分都可以作为频率的函数被列出或者绘制出。每一个失真源的贡献也可以在选定的频率点上被打印出来。

\section{SPICE的设计}
SPICE中包含的电路元件和分析能力反映了几年中一段时期内一个特定的集成电路设计组的需求。很明显,还有很多仿真能力没有被加入SPICE中。例如,评估小信号AC转移函数的极点和零点的能力就没有被包含进SPICE中,因为我们的SLIC程序\cite{ref-14,ref-15}已经有了该特征。因为时域稳态分析\cite{ref-16,ref-17,ref-18}在我们的SINC程序中\footnote{SINC是一个非线性瞬态分析程序,由University of Californian,Berkeley的电子研究实验室的S. P. Fan教授开发。SINC程序从TIME程序进化来\cite{ref-19}。}已经实现,该能力也没有被包含进SPICE中。

很多仿真程序包含了统计分析\cite{ref-20,ref-21,ref-22},该分析评估随机电路参数采样下的电路性能。另外,很多程序已经被开发出来使用仿真程序来自动化设计优化\cite{ref-23,ref-24,ref-25,ref-26,ref-27}。对于统计分析或者自动化设计,没有新的电路分析能力被要求。然而,因为这两种能力都要求大量的重复性电路分析,所以分析程序的效率无论在统计分析程序还是自动化设计程序的开发中都是最重要的。

仿真程序的设计受程序使用模式的指导原则影响。影响SPICE程序设计的指导原则在本论文中被采纳为一种比较的方法。这些指导原则或多或少存在它们顺序的优先级:使用的便捷性,有效性,简单性,和通用性。

很清楚,一个仿真程序用起来必须要简单。电路设计人员总的来说不是计算机编程人员,所以一个仿真程序的使用应该对程序技术有着最低知识储备的要求。毕竟,仿真程序理想上等同于实验室工作台。

仿真程序的效率决定了计算机仿真的钱的开销。如果仿真程序使用起来不贵,那么计算机资源,比如执行时间,中央内存,和硬盘存储,应该最小化。而且,任何电路设计项目要求对不同参数值,不同温度,和等等进行重复的电路分析。这种重复分析的需求对统计分析或者自动化设计的情况也同样适用。因此,仿真程序必须,作为一项非常高的优先级,要求最少数量的计算机资源来执行电路仿真。

仿真程序的架构,以及在仿真程序中使用的计算方法应该尽可能简单地确保在修改,增强,和支持的过程中容易完成。成熟的和复杂的编程方法因此只有在它们能够在程序效率或者使用方便上产生相当增长的情况下才应该被使用。程序中通用性的需求通常会与简便性和有效性的需求冲突。对大多数部分来说,SPICE程序在通用性设计方面的优先级比在简便性或者有效性方面的要低的多。

本论文其余章节的话题聚焦于电路理论和在电路仿真程序的分析部分占中心位置的数值分析。SPICE中使用的算法会被详尽提出,而这些方法的性能会被与其他可行技术的性能做比较。因为没有一条“正确”开发仿真程序的道路,所以SPICE程序中分析方法的选择和实现是以本章前面提到的指导原则为基础来评判的。很清楚,不同的仿真需求和优先级经常会产生出对不同计算算法的选择或者实现。

本论文的第\ref{chap:2}章给出了电路仿真程序的概览以及在分析子程序中使用的计算方法。第\ref{chap:3}章提出了方程构造的方法,这些方程构造通过电路方程系统被用来代表电路。在第\ref{chap:4}章,描述了求解线性方程系统的方法。第\ref{chap:5}章提出和比较了迭代求解非线性方程系统的方法。最后,第\ref{chap:6}章提出了数值积分方法,这些方法对执行瞬态分析是重要的。在第\ref{chap:7}章,给出了SPICE中实现的算法的整体总结,以及未来工作的一些建议。

附录\ref{App:1}详细给出了被用来比较分析算法的电路案例。附录\ref{App:2}描述了SPICE程序中已采用的元件模型。SPICE程序的输入语法在附录\ref{App:3}中被详细讨论。附录\ref{App:4}描述了SPICE2程序的实现。最后,SPICE2程序的全部列表在附录\ref{App:5}中给出。
