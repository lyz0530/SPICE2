\chapter*{摘要}
\markboth{Preface}{摘要}
目前集成电路的规模和复杂度已经增长到实质上离不开计算机辅助的程度。表征电路电气性能的电路仿真程序在电路设计过程中是一种重要的计算机辅助措施。对精准和有效的电路仿真的需求促进了很多电路仿真程序的开发,以及相关数值方法的进步。

本论文有两个目的。第一,详细介绍了电路仿真程序中的必要成分,数值方法,并且对这些方法进行了描述和比较。第二,SPICE1和SPICE2的理论和实践方面被记录。SPICE是目前被大部分电子工业使用的几个电路仿真程序中的一个。SPICE主要是由作者开发的。SPICE中使用的数值方法,当然,是很多研究人员的结果。

本论文从对SPICE程序的功能性描述开始。电路仿真整个工作的展示被划分成了各个议题,方程形成,线性方程解,非线性方程解,和数值积分。本论文的主体是在每一个这些一般性的议题中,对那些可行的方法进行比较。这些比较是在对典型电子电路仿真问题的方法性能的评估基础上展开的。

两个SPICE版本中使用的数值方法都进行了详细地展示。这些方法是根据本论文中简介章节提出的指导原则选择的。不同指导原则可能会导致不同方法的开发。SPICE的广泛使用,然而,表明在SPICE中使用的算法对很宽范围内的实际电路仿真问题是可行的。