\chapter{总结}
\label{chap:7}
很少有通用的数值方法不经过一些修正就能被用到一个特定的应用上。SPICE程序中使用的和本论文中说明的这些算法是在电路仿真的特定背景下,对一些已经存在的数值方法进行小心实现,修正,和比较的结果。本论文尽可能尝试分开比较用在电路仿真程序中的大部分数值技术。这些比较的目标当然是为了把那些“最好”的方法结合进程序中,以使得电子电路的仿真在程序用户交互最少的情况下可以精确高效地进行。

在本论文的第\ref{chap:2}章,电路仿真任务被分成了方程构建的算法方法,线性方程求解,非线性方程求解,和数值积分。幸运地是,这四个领域彼此相对独立,也就是在一个领域对一种特定实现的选择对剩余领域内算法的选择没有什么影响。这种独立使得在第\ref{chap:3}章,第\ref{chap:4}章,第\ref{chap:5}章和第\ref{chap:6}章这些给定的主题内能够对方法进行分开比较.每个章节的重要的结果总结如下。

可行的很多方程构建方法可以被分成三类:节点分析,混合分析,和稀疏表格分析。如果假定做出一些修正以使得可以处理电压定义支路的话,那么节点分析被发现与其他的方法有大致上相同的普适性。对于测试的样品电路,修正的节点分析方法产生的电路方程系统需要的计算代价远远小于混合分析需要的。而且,节点方法实现起来比混合方法或者稀疏表格分析要简单的多。

在SPICE2程序中实现了修正的节点分析(MNA)。这个方法提供与其他方法一样的普适性,而且它产生的近对称方程系统的求解需要的计算代价和最简单的节点方法差不多。然而,在SPICE2中实现MNA遇到了两个问题。第一个,在一些案例中出现了数值病态的问题。第二个,MNA处理接地电压源的方式不是最优的。MNA的一种修正,叫作MNA2,虽然消除了这些问题但是带来了不对称的系数矩阵,和在稀疏矩阵技术中与之相称的负担增加。因此,在SPICE2中保留了原始的MNA方法。下一步研究应该得出一种修正的节点分析方法,在不破坏稀疏矩阵对称性的前提下消除前面提到的问题。

实现线性方程求解方法被证明是相当简单的。因为SPICE在有着大的字长度的CDC 6400上实现,所以线性方程求解中的舍入误差通常是可以忽略的,而且求解方法可以基于计算代价而不是精度来优化。LU分解的直接消去方法,因此,是线性求解方法的合理选择。SPICE中包含的LU分解的稀疏矩阵实现在第\ref{chap:4}章中进行了详细介绍。

为了最小化方程求解中的计算代价,有必要对电路方程重新排序,以最小化填元的数量,并且因此可以在求解过程中保持源方程的稀疏性。第\ref{chap:4}章令人惊奇的结果是所有提出的重排序算法得到了大致相同的系数矩阵规模和运算次数。这一结果对典型电路和随机生成的系数矩阵的数值样品都成立。因为更复杂的重排序算法也没有提升,所以在SPICE2程序中采用了简单的Markowitz方法,它在决定重排序时需要最少的cpu时间。

编码线性电路求解函数的特殊方法对效率有着深刻的影响。发现汇编语言求解函数要求的计算代价是FORTRAN函数的一半。生成可执行机器代码在执行时间上能产生另外的减少,但需要额外的内存开销。对AC分析,内存上的巨量增加与执行时间上获得的适量节省不匹配。在SPICE2中,DC和瞬态方程是用可执行机器代码求解的,而AC方程是用一个汇编子函数求解的。

Newton-Raphson算法是求解非线性电路的不二选择。然而,除非做出一些合适的修正,否则Newton-Raphson算法不能被应用到电路仿真问题中。如第\ref{chap:5}章中详细介绍的,电路仿真程序中已经使用过了大量的修正的Newton-Raphson算法。这些算法自然分为了三组,simple-limiting方法,error-reduction方法,和source-stepping方法。这三种方法的对比表明simple-limiting方法是最有效的,而且拥有令人满意的收敛特性。在simple-limiting方法中,来自Colon的方法被证明是最可靠的,也是最高效的。出于这个原因,SPICE2采用了Colon的simple-limiting方法。

瞬态分析需要数值积分算法和动态变化积分时间步长的方法来保持合理的求解精度。有很多应用到电路仿真的积分方法。在第\ref{chap:6}章做的一个初步调查表明积分技术的详细对比可以被限制为三种方法:梯形规则算法,Gear-2方法,和Gear的变阶算法。在这三者中,Gear变阶方法,同时动态变化积分阶数和时间步长,看起来最有希望,因为为了维持一个特定的阶数使用更高阶方法理论上应该在积分中能带来更少的时间点数。然而,变化阶数的机制导致额外的负担,恶化期望的计算节省,而且在第\ref{chap:6}章中详细介绍的梯形实现确实对所有测试的案例都能带来最少的计算代价。因此,梯形积分,带截断误差时间步长控制,在SPICE2程序中被采纳。

在仿真程序中采用的分析技术严重依赖于整个程序要求的能力。电路在输入被描述的方式,例如,对分析方法的选择施加了明显的约束。该点通过对ASTAP程序和SPICE程序的对比进行了说明。SPICE中的支路关系是预定义的,这样这些支路关系的导数,在Newton-Raphson线性化中需要的,就容易来评估。在ASTAP程序中,支路关系是以算术表达式或者表格的方式提供的,而这些支路关系的导数必须被数值地评估。很清楚,对SPICE开发的最优的线性化方案与为ASTAP开发的最优的线性化仿真有很大的不同。在本章总结的结论,因此,假定了程序的输入和架构是SPICE中采用的类型。

任何仿真程序的最终效果是通过它对电路设计过程的有用性来衡量的。从这个角度看,SPICE表现地令人赞赏。目前SPICE程序已经被超过100个大学和半导体公司使用,而该使用的舆论意见是SPICE是一个有价值的设计工具。当然,还有很多不适合SPICE的仿真问题。这些问题中值得注意的例子是近正弦的非线性电路,比如振荡器和混合器。无法接受任意的支路关系作为输入也阻碍了SPICE在一些特殊目的的应用上的使用。下一步的研究应该提供分析近正弦电路的新技术。SPICE的另外开发也应该在不破坏当前程序的简单性和有效性的前提下提供一种更通用的输入描述。