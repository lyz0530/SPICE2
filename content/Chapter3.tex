\chapter{方程建立}
\label{chap:3}
电子电路的计算机仿真包括电路数学模型的数值分析。数学模型,因此,在任何电路分析中都是排在第一的事情。物理电路包含电路元件的互联。电路的数学模型,另一方面,是一个方程系统。这些方程的解,对于指定的特定案例,仿真该电路的指定的电气特性。

为了求解描述电路的方程系统,每一个电路元件首先应该用数学的术语来被建模。这个建模步骤要求每一个元件用一个理想网络分支的互联来表示。每一个分支,反过来,用一个数学关系指定,该数学关系把支路的电压或者电流定义为未知电路变量的函数。一个电阻,例如,会被通过用一个单电阻值联系起来的支路电流和支路电压的线性方程定义;名义上,支路电流,$I_R$,与支路电压,$V_R$,通过方程$I_R R = V_R$,这里R是电阻的值,联系起来。

建模了电子电路的方程系统必须满足两个约束。第一,每个支路关系必须在时间上的每个时刻被实时满足。第二,Kirchoff的拓扑法则,KCL和KVL,必须在时间上的每个时刻被实时满足。

在时间的每个时刻每个支路关系的满足的约束预示着形如下面的代数差分方程系统

\begin{equation}
    B(x,\dot{x},t)=0
    \label{eq:3.1}
\end{equation}
这里向量x包含着支路电压和电流,$\dot{x}$是x的时间导数,而B是,通常来讲,一个非线性算子。

Kirchoff拓扑规则,KCL和KVL,预示着下面的方程系统
\begin{equation}
    T x = 0
    \label{eq:3.2}
\end{equation}
这里矩阵T包含的系数是+1,-1,或者0。

如果一个特殊的电路包含b条支路,那么(\ref{eq:3.1})就是b个方程系统,而(\ref{eq:3.2})也是b个方程系统\cite{ref-7}。(\ref{eq:3.1})和(\ref{eq:3.2})的直接组合会产生如下形式的2b个代数差分方程
\begin{equation}
    F(x,\dot{x},t)=0
    \label{eq:3.3}
\end{equation}
这里F是,通常来讲,一个非线性算子。

幸运的是,没有必要把所有的支路电压和电流包含进向量x中。实际上,x会包含能产生一个独立方程系统的电路变量的任意组合。未知向量的大范围可选择性导致各种不同方程构造方法的可行性。

例如,一个含有100个节点,100个晶体管的电路可以用一个含有700条支路,100个节点的网络模型来建模。(\ref{eq:3.1})和(\ref{eq:3.2})的直接结合会产生一个含有1400个方程的系统。然而,如果构造方程时采用了节点分析法,产生的数学模型会只包含100个方程。数学模型的复杂性,因此,受用来形成该模型的方法的控制。

本章提出了不同的方程构造的方法。首先,理想的支路关系会被提出。之后,Kirchoff定律暗指的约束会用(\ref{eq:3.2})的形式形成。一旦(\ref{eq:3.1})和(\ref{eq:3.2})建立,Cutset分析和Loop分析的主方法会被介绍。这些方法都给可以被包含进电路的支路关系的类型施加了限制。实际的构造方法,因此,是这两种方法的组合。

目前可行的仿真程序采用了三种不同的构造方法中的一种。节点分析法的一种修正被用在SPICE,CANCER\cite{ref-2,ref-3},SLIC\cite{ref-14,ref-15},TIME\cite{ref-19},TRAC\cite{ref-39},和ECAP\cite{ref-40}中。第二种受欢迎的构造技术,混合分析,被用在ASTAP\cite{ref-28,ref-29},ECAP2\cite{ref-31,ref-32},NET2\cite{ref-33},CIRCUS2\cite{ref-34},CIRPAC\cite{ref-41,ref-42},SCEPTRE\cite{ref-43},NET\cite{ref-44},和CIRCUS\cite{ref-45}。第三种构造算法,在仿真程序中还没有看到被广泛使用,是Hachtel的稀疏表格构造法,等等\cite{ref-46}。

在实际中,电路方程从来不会用(\ref{eq:3.3})的形式构造。相反,在DC分析的每一次迭代中,AC分析的每一个频率点上,以及瞬态分析的每一个时间点的每一次迭代中,支路关系(\ref{eq:3.1})会被用等价的线性方程系统取代。当把方程构造算法应用到线性方程的该等价系统上时,产生的电路方程系统是具有如下形式的方程线性系统
\begin{equation}
    A x = b
    \label{eq:3.4}
\end{equation}
这里A是系数矩阵,而b是激励向量。

因为对于大多数构造技术来说系数矩阵倾向于包含很少的非零系数,所以求解(\ref{eq:3.4})的方法会利用系数矩阵的稀疏性。构造算法因此应该产生一个含有最少非零元数目的系数矩阵。在理论上,至少,A的维度没有包含在A中的非零元总数那么重要。

\section{支路关系}
为了给电路建立数学模型,每一个电路元件被定义成理想支路之间的相互连接。给定电路元件的建模是由,当然,该元件的物理性质决定。在本论文的附录\ref{App:2}中给出了SPICE元件的模型。

支路电压和支路电流的极性根据“相对参考方向”的概念\cite{ref-7}被随意指定。这个概念在图\ref{图3.1}中进行了说明。简单来说,如果支路的正节点比负节点处于一个更高的电势,那么支路电压就是正的;如果它从正节点,流经支路,到负节点,那么直流电流就是正的。

\begin{figure}[htbp]
\small
    \centering
    \includegraphics[width=0.7\textwidth]{figure/Chapter3/图3.1.png}
    \caption{支路极性定义。}
    \label{图3.1}
\end{figure}

支路关系的四种常见类型在表\ref{表3.1}给出。这个表中的向量x包含所有的支路电压和支路电流。如果支路电流是基于电路参数和电路变量定义的,那么支路就是电流定义的。如果支路电压是基于电路参数和电路变量定义的,那么支路就是电压定义的。例如,电压源和电感是电压定义的,这里电流源和电容是电流定义的支路。有限、非零阻值的理想电阻,可以被建模为电压定义的支路或者电流定义的支路。

\begin{figure}[htbp]
\small
    \centering
    \includegraphics[width=0.7\textwidth]{figure/Chapter3/表3.1.png}
    \caption{支路关系的四种常见类型。}
    \label{表3.1}
\end{figure}

通常,支路关系基于支路电压来定义支路电流,反之亦然。受控支路,另一方面,由决定于其他支路变量的支路关系定义。如果一条受控支路依赖于其他支路电压,那么该支路就是电压控制的。如果该支路依赖于其他的支路电流,那么该支路就是电流控制的。一条支路有可能既是电压控制的,也是电流控制的。

\section{Kirchoff拓扑规则}
除了支路关系施加的约束外,电路方程必须满足Kirchoff的拓扑定律\cite{ref-7}:
\begin{enumerate}
    \item (KCL) - 流经网络的每一个割集的电流的代数和必须在每一个时刻都等于0。
    \item (KVL) - 网络的每一个环路的电压的代数和必须在每一个时刻都等于0。
\end{enumerate}
为了把KCL和KVL转化成(\ref{eq:3.2})的形式,选择电路的树。树是连接每一个节点的支路的子集,而且不包含任何环。树中包含的网络支路被称作树支路。排除出树的支路被称作树连接,树弦,或者树补支路。

每一条树支路独特定义一个基本的割集,这就是,一个包含了给定树支路和一个或者多个树连接的割集。一个网络的任何一棵树包含n-1条支路\cite{ref-7};因此,共有n-1个基本的割集。这n-1个基本的割集方程通过对每个基本的割集施加KCL得到:
\begin{equation}
    [I_{n-1}|F][\frac{I_T}{I_L}]=0
    \label{eq:3.5}
\end{equation}

矩阵$I_{n-1}$是一个$(n-1)\times(n-1)$的单位阵,F是维度$(n-1)\times(b-n+1)$的基本割集矩阵,$I_T$是树支路电流的向量,而$I_L$是树连接电流的向量。

每一个树连接单独定义一个基本的环,这就是说,一个只含有给定树连接和一个或者多个树支路的环。电路的任意一棵树必然含有b-n+1个连接,因为它含有n-1条树支路;因此,共有b-n+1个基本的环。这b-n+1个基本的环方程通过对每一个基本的环施加KVL得到

\begin{equation}
    [I_{b-n+1}|-F^T][\frac{V_L}{V_T}]=0
    \label{eq:3.6}
\end{equation}
这里$I_{b-n+1}$是维度为$(b-n+1)\times(b_n+1)$的单位阵,$-F^T$是维度为$(b-n+1)\times(n-1)$的基本环矩阵。$V_L$是树连接电压的向量,而$V_T$是树支路电压的向量。

方程(\ref{eq:3.5})和(\ref{eq:3.6})一起等价于(\ref{eq:3.2})。因此,矩阵T通过下面的方程定义
\begin{equation}
    T=\begin{bmatrix}
I_{n-1} &F  \\
 -F^T& I_{b-n+1} \\
\end{bmatrix}
\label{eq:3.7}
\end{equation}
T的决定没有吸引力,实际上,因为必须选择一棵树,而矩阵F必须被推导出来。幸运地是,有方程构造方法不要求矩阵T。

\section{割集分析}
割集分析的构造方法使用n-1条树支路电压作为未知的电路变量。基本的割集矩阵,Q,通过下面的方程定义
\begin{equation}
    Q=[I_{n-1}|F]
    \label{eq:3.8}
\end{equation}
方程(\ref{eq:3.5})和(\ref{eq:3.6})简化后得到
\begin{equation}
    Qi_b=0
    \label{eq:3.9}
\end{equation}
\begin{equation}
    v_b = Q^Tv_t
    \label{eq:3.10}
\end{equation}
割集分析是从所有的支路关系是由电流定义的和电压控制的假定下得到的。因此,在割集分析中是不容许存在电压源或者电感的,而且电流控制的源也是不被容许的。为了简化计算,只考虑线性支路关系的情况。在这些假设下,(\ref{eq:3.1})中的支路关系的一般系统可以被简化为
\begin{equation}
    Bv_b +j_b = i_b
    \label{eq:3.11}
\end{equation}
矩阵B是支路关系的系数矩阵,而向量$j_b$是独立电流源值的向量。现在,(\ref{eq:3.9}),(\ref{eq:3.10})和(\ref{eq:3.11})被结合起来以得到网络方程
\begin{equation}
    QBQ^Tv_t = -Qj_b
    \label{eq:3.12}
\end{equation}

\section{环路分析}
环路分析是割集分析的对偶形式。b-n+1支树连接电流被用作未知向量,而支路关系被限制为电压定义的和电流控制的。基本的环路矩阵,L,定义为\footnote{在很多文献中\cite{ref-7},基本的环路矩阵被表示为B。然而,本章里的B矩阵表示支路关系,所以L被用来表示基本的环路矩阵。}
\begin{equation}
    L=[F_t|I_{b+n-1}]
    \label{eq:3.13}
\end{equation}
方程(\ref{eq:3.5})和(\ref{eq:3.6})随后可以被重新声明为
\begin{equation}
    i_b = L^Ti_L
    \label{eq:3.14}
\end{equation}
\begin{equation}
    Lv_b= 0
    \label{eq:3.15}
\end{equation}
该支路关系被限制为(\ref{eq:3.11})的对偶形式,也就是
\begin{equation}
    Bi_b +e_b = v_b
    \label{eq:3.16}
\end{equation}
这里$e_b$是独立电压源值的向量。因此,结合(\ref{eq:3.14}),(\ref{eq:3.15})和(\ref{eq:3.16})得到环路方程的系统
\begin{equation}
    LBL^Ti_b = -Le_b
    \label{eq:3.17}
\end{equation}

\section{节点分析}
方程构造的节点方法是割集分析中的一种不要求选择网络树的特殊情况。节点分析通过使用n-1个非参考节点电压作为未知量。b条支路电压,$v_b$,和n-1个非参考节点电压,$v_n$,通过节点关联矩阵相关起来如下:
\begin{equation}
    v_b = A^Tv_n
    \label{eq:3.18}
\end{equation}
节点关联矩阵由下面的系数定义
\begin{equation*}
 a_{ij}=\left\{\begin{matrix}
0 & \text{如果节点i和支路j不相连} \\
+1 & \text{如果节点i是支路j的正节点} \\
-1 & \text{如果节点i是支路j的负节点} \\
\end{matrix}\right.   
\end{equation*}
节点关联矩阵是描述不依赖选择网络树的网络拓扑的一种简洁办法。实际上,A可以直接由输入数据定义。定义F的大多数方法从节点关联矩阵开始,而且通过诸如Gauss-Jordan消去\cite{ref-29}的方法从A获得F。

可以证明\cite{ref-7},节点关联矩阵也能直接表示KCL如下:
\begin{equation}
    Ai_b = 0
    \label{eq:3.19}
\end{equation}
很清楚,(\ref{eq:3.17})和(\ref{eq:3.9})相似,而(\ref{eq:3.18})和(\ref{eq:3.10})相似。结合(\ref{eq:3.11}),(\ref{eq:3.18}),和(\ref{eq:3.19})产生节点方程系统:
\begin{equation}
    ABA^Tv_n = -Aj_b
    \label{eq:3.20}
\end{equation}
方程(\ref{eq:3.20})随后可以用简化的标识来表示得到
\begin{equation}
    Yv_n = j
    \label{eq:3.21}
\end{equation}
这里Y是节点导纳矩阵,$v_n$是节点电压向量,而j是节点电流激励向量。

与割集分析一样,节点分析的最简形式只能应用到电流定义的和电压控制的支路关系上。支路关系的其他类型必须通过插入串联电阻转化为电流定义的电压控制的支路关系。电压源,例如,通过图\ref{图3.2}中所示的Norton等效支路关系被取代。
\begin{figure}[htbp]
\small
    \centering
    \includegraphics[width=0.7\textwidth]{figure/Chapter3/图3.2.png}
    \caption{独立电压源转化。}
    \label{图3.2}
\end{figure}
节点导纳矩阵可以用一种非常有效的方式构建;也就是说,每一个支路关系被加到与支路正节点对应的Y的行,从与支路负节点对应的Y的行减去。例如,考虑一个连接在节点i和节点j之间的电阻,通过支路关系定义如下
\begin{equation}
    I_b = \frac{1}{R}(v_i - v_j)
    \label{eq:3.22}
\end{equation}
该支路通过把$1/R$加到$y_{ii}$和$y_{jj}$的系数上,从$y_{ij}$和$y_{ji}$那里减去$1/R$来调整节点导纳矩阵。

为了简化节点分析的实现,参考节点被包含进了(\ref{eq:3.21}),作为,叫作,第一个节点。节点方程随后就可以在不需要区分接地支路和非接地支路的需要下被组装。在(\ref{eq:3.21})的解中,与参考节点相关的Y的行和列被忽略了。矩阵Y,带着参考节点的行和列的加入,被参照为不定的导纳矩阵\cite{ref-7}。

\section{带电压源的节点分析}
节点分析的最严重的缺陷是没有处理电压定义支路的能力。然而,非常多种类的集成电路可以通过添加提供接地电压源来与节点技术兼容。这里提到的方法在SLIC\cite{ref-14,ref-15}和TIME\cite{ref-19}中实现。算法中加入接地电压源消除了与串联电阻发生的病态条件问题,当然也会带来更少的计算代价。

网络模型中的接地电压源的存在把未知的节点电压的数目减少了一。通过把节点方程划分进源方程和约减后的方程,接地电压源被包含了进来。如果存在n个节点和$n_v$个接地电压源,节点会被重新编号,这样节点1是参考节点,节点2到$n_v+1$是与接地电压源连接的节点,而节点$n_v+2$到n是剩余的电路节点。节点方程的划分集合用之前描述过的方法组合得到
\begin{equation}
    \begin{bmatrix}
 I_v\\
 0
\end{bmatrix}+
\begin{bmatrix}
y_{ss} & y_{sr} \\
y_{rs} & y_{rr} \\
\end{bmatrix} \begin{bmatrix}
v_s \\
v_r
\end{bmatrix}=
\begin{bmatrix}
J_s \\
J_r
\end{bmatrix} 
\label{eq:3.23}
\end{equation}
向量$I_v$包含电压源支路电流;如果(\ref{eq:3.23})是KCL的一个正确的表示,那么这个向量就必须被包含进来。方程(\ref{eq:3.23})被划分这样$V_s$就是已知的,也就是,
\begin{equation}
    V_s = E
    \label{eq:3.24}
\end{equation}
这里$E = (0,E_1,E_2...)^T$,$E_1$是第一个接地电压源的值,$E_2$是第二个接地电压源的值,以此类推。向量$V_r$是由下面方程定义的约减的节点电压的未知集合
\begin{equation}
    y_{rr}V_{r} = J_r - y_{rs}E
    \label{eq:3.25}
\end{equation}
接地电压源电流随后由下面的方程定义
\begin{equation}
    I_V = J_S - y_{ss}V_{s}-y_{sr}V_{r}
    \label{eq:3.26}
\end{equation}
DC分析中的浮动电压源和电感,可以被包含进节点分析中,不过要有额外的矩阵操作开销。节点方程用前面的相同方式被划分。然而,虽然$V_s$不再是已知的,但是将依赖于$V_r$。这暗示着,对每一个源,$y_{ss}$和$y_{rs}$中与正节点对应的列必须被加入到$y_{sr}$和$y_{rr}$的与负支路节点对应的列;另外,$y_{ss}$和$y_{sr}$中的与支路正节点对应的行必须被加入到$y_{rs}$和$y_{rr}$中与支路负节点对应的行中。向量$V_r$随后像之前通过求解下面方程得到
\begin{equation}
    \hat{y_{rr}}V_r = J_r - \hat{y_{rs}}E
    \label{eq:3.27}
\end{equation}
这里$\hat{y_{rr}}$和$\hat{y_{rs}}$表示子矩阵,这些子矩阵通过刚刚描述过的行和列的操作得到。向量$V_s$由$V_r$和E决定。最后,如果期望的话,电压源电流可以像之前从(\ref{eq:3.26})得到。

CANCER\cite{ref-2}和SPICE的第一个版本采用了带浮动电压源加入的节点分析构造。在DC分析中,电感被当作0值电压源。这种实现的缺点是如果电路包含电感,DC分析的矩阵结构与AC或者瞬态分析的不同。而且,要求的矩阵操作不是期望的,因为它们会给算法引入额外的记账。然而,如果仿真的电路几乎不含有电感或者浮动电压源,就像在IC设计中经常遇到的情况,那么CANCER方程构造就是简单的和有效的。

\section{修正节点分析}
电压定义支路和电流控制支路的加入通过C. Ho等人\cite{ref-47}开发的修正节点分析(MNA)方法用一种非常简单的形式满足了。如果电路含有n个节点和$n_v$个电压定义的支路,那么n-1个非参考节点电压和$n_v$个电压定义的支路电流就变成了未知的电路变量。修正的节点方程组通过在每个非参考节点上施加KCL和包含$n_v$个电压定义的支路关系得到。方程组就成了如下的形式
\begin{equation}
    \begin{bmatrix}
    Y & B\\
    C & D
    \end{bmatrix}\begin{bmatrix}
    V_n \\
    I_b
    \end{bmatrix}=\begin{bmatrix}
    J \\
    E
    \end{bmatrix}
    \label{eq:3.28}
\end{equation}
这里$V_n$是节点电压向量,$I_v$是电压定义支路电流向量,J是节点电流激励向量,而E是电压定义支路电压向量。因此,Y和J可以用(\ref{eq:3.21})中的方式构建。如果在网络中没有电流控制的,电流定义的支路关系,那么B矩阵的整数系数$b_{ij}$通过下面的方法定义
\begin{equation*}
 b_{ij}=\left\{\begin{matrix}
0 & \text{如果节点i和支路j不相连} \\
+1 & \text{如果节点i是支路j的正节点} \\
-1 & \text{如果节点i是支路j的负节点} \\
\end{matrix}\right.   
\end{equation*}
如果该网络含有电流控制的,电流定义的支路关系,那么B矩阵也有会定义这些关系的耦合系数。

\begin{figure}[htbp]
\small
    \centering
    \includegraphics[width=0.7\textwidth]{figure/Chapter3/图3.3.png}
    \caption{示例电路。}
    \label{图3.3}
\end{figure}

C和D矩阵以及向量E一起包含着$n_v$个电压定义支路的支路关系。在没有电压定义或者电流控制的支路的案例中,(\ref{eq:3.28}),简化为简单的节点方程(\ref{eq:3.21})。对于图\ref{图3.3}中所示的电路,MNA电路方程是
\begin{equation}
    \begin{bmatrix}
    1/R_2 & -1/R_2 & 0 & 1 \\
    -1/R_2 & 1/R_2 + 1/R_3 + C_4\frac{d}{dt} + C_5\frac{d}{dt} & -C_5 \frac{d}{dt} & 0 \\
    0 & g_6 - C_5\frac{d}{dt} & C_5\frac{d}{dt} + 1/R_7 & 0 \\
    1 & 0 & 0 & 0
    \end{bmatrix}\begin{bmatrix}
    v_1 \\
    V_2 \\
    V_3 \\
    I_1
    \end{bmatrix} = \begin{bmatrix}
    0 \\
    0 \\
    0 \\
    E
    \end{bmatrix}
    \label{eq:3.29}
\end{equation}
(\ref{eq:3.29})对角线上的0系数的存在不是一个计算问题,因为矩阵结构是这样的,0对角元在求解的过程中总会变成非零元。然而,0对角项的可能性确实意味着分解的顺序得是这样的,包含0对角系数的行在填元发生后才能被处理。

\begin{figure}[htbp]
\small
    \centering
    \includegraphics[width=0.7\textwidth]{figure/Chapter3/图3.4.png}
    \caption{用来说明MNA中病态条件的电路。}
    \label{图3.4}
\end{figure}

然而,MNA构造的过程存在计算问题。MNA构造遇到的病态条件通过图\ref{图3.4}中所示的电路说明。对于DC分析,该电路的MNA方程是
\begin{equation}
    \begin{bmatrix}
    1/R_2 & -1/R_2 & 0 & 1 \\
    -1/R_2 & 1/R_2 + 1/R_4  & -1/R_4 & 0 \\
    0 & -1/R_4 & 1/R_4 & 0 \\
    1 & 0 & 0 & 0
    \end{bmatrix}\begin{bmatrix}
    v_1 \\
    V_2 \\
    V_3 \\
    I_1
    \end{bmatrix} = \begin{bmatrix}
    0 \\
    0 \\
    0 \\
    E
    \end{bmatrix}
    \label{eq:3.30}
\end{equation}
不管方程用哪一种顺序被消去,节点电压对角项中的一个会变成0,而求解会失败。

这个病态条件问题,一旦被识别出来,可以用一种直接的修正来求解。在(\ref{eq:3.30})的例子中,如果第一行和最后一行互换,那么可以得到下面的方程系统。
\begin{equation}
    \begin{bmatrix}
    1 & 0 & 0 & 0 \\
    -1/R_2 & 1/R_2 + 1/R_4  & -1/R_4 & 0 \\
    0 & -1/R_4 & 1/R_4 & 0 \\
    1/R_2 & -1/R_2 & 0 & 1
    \end{bmatrix}\begin{bmatrix}
    v_1 \\
    V_2 \\
    V_3 \\
    I_1
    \end{bmatrix} = \begin{bmatrix}
    E \\
    0 \\
    0 \\
    0
    \end{bmatrix}
    \label{eq:3.31}
\end{equation}
这种行的互换可以推广如下。对于网络中每一条电压定义的支路,在未知向量中就会有一个支路电流变量。如果电压定义的支路没有冲突的节点,那么通过把每个电压定义的电流行与该支路正节点对应的行交换就可以完成互换。对于电压定义的支路有冲突节点的情况,有必要在电压定义支路电流和会包含几个电压定义支路负节点的网络节点之间建立一种映射。这种映射通常都可以被构建,除非网络含有环状的电压定义支路;在这种情况下,对KVL的一种可能的违反会由网络拓扑表明。

行的交换不是计算地执行。因为每一个矩阵系数是通过一个整数指针系统访问的,所以行的交换是通过调整指针系统完成的。因此,在花费少量的额外建立代价之后,由于行交换引起的计算代价在量上没有差别。

不幸的是,行必要的交换破坏了节点方程近似对称的结构。在之前的节点实现中,Y矩阵在结构上是近似对称的,而且很少额外的0值项会通过假定对称被引入。关于对称的假定完全能够把稀疏矩阵指针结构的规模降低一半,因而是有很高的期许。MNA2算法增加的通用性对指针结构增加的规模是否值得值得商榷。

\section{节点分析方法的对比}
上面提到的三种节点方法在SPICE中都实现了,并且用本论文附录\ref{App:1}中描述的十个标准的基准电路进行了测试。这三种方法各自的方程的数量,非零元矩阵系数的数量,以及运算操作的数量都在表\ref{表3.2}中给出。在这个表中,NODAL指在CANCER中使用的节点方法,MNA1指由C. Ho等人\cite{ref-47}原始提出的MNA方法,而MNA2指带行交换的MNA方法。表\ref{表3.2}中引用的操作的数量是形如$A=A-B \times C$或者$A=A/B$这种形式的操作。在这10个电路中提出的唯一的一种电压定义的支路是模拟了电源供应和电路输入的接地的电压源。
\begin{figure}[htbp]
\small
    \centering
    \includegraphics[width=0.7\textwidth]{figure/Chapter3/表3.2.png}
    \caption{节点分析方法的对比。}
    \label{表3.2}
\end{figure}

从这个数据可以清楚看到当和NODAL方法或者MNA2方法对比时,构造MNA1方法要求更多的计算代价。对于UA741电路,作为一个例子,为了求解电路方程系统,MNA1方法构造的系数矩阵需要额外的17\%的非零项,和57\%的计算操作。

为了进一步论证这三种构造方法中的区别,对这十个带接地电压源的电路重新进行了测试,这里接地电压源用不接地电压源和一个小电阻的串联来代替。该测试结果如表\ref{表3.3}所示。矩阵规模,和操作数量,在NODAL方法和MNA2方法上有了大量的增长,而且这些数字对MNA1方法也是近似相同的。对UA741电路这个相同的例子,MNA1方法现在产生出的系数矩阵比NODAL构造出的结果小11\%,而对MNA1方程的计算代价比NODAL方程的计算代价小5\%。然而,MNA2方法的矩阵比MNA1方法的小4\%,而MNA2方程的计算代价比MNA1方法的代价小18\%。

\begin{figure}[htbp]
\small
    \centering
    \includegraphics[width=0.7\textwidth]{figure/Chapter3/表3.3.png}
    \caption{带浮动源的节点分析方法的对比。}
    \label{表3.3}
\end{figure}

\section{混合分析方法}
混合分析方法基于支路电压和支路电流形成电路方程。所有这些方法都要求网络树的选择;电路方程随后会从基本的割集方程和基本的环路方程推导得来。混合分析方法包含用在CIRPAC\cite{ref-41,ref-42},SCEPTRE\cite{ref-43},NET\cite{ref-44},和CIRCUS\cite{ref-45}中的状态变量分析公式。ECAP2\cite{ref-31,ref-32}中使用的树连接公式\cite{ref-48}和构造方法也是混合方法,就像ASTAP\cite{ref-28,ref-29}中使用的修正表格法。

网络树选择是构造算法中非常关键的一部分内容。Kirchoff定律在树选择的过程中施加了特定的约束。特别地,独立电压源不能是树连接,而独立电流源不能是树支路。得到的电路方程的条件有多好依赖于选择的树。大多数混合方法根据“ECRLJ”过程规则选择树。简言之,独立电压源(E支路)通常变为树支路,而独立电流源(J支路)通常变为树连接。电容(C支路)只要有可能会变成树支路,而电感(L支路)只要有可能会变成树连接。如果电路不包含电容性环路或者电感性割集,那么所有电容会被包含进树中,而所有电感会被从树中排除。电阻(R支路)要不变成树支路,要不变成树连接。

树连接公式\cite{ref-47}是“ECRLJ”过程规则的一个值得注意的例外。用提出的这个公式,树支路根据它们等效的电导值来被选择,大的电导值会被给予最高的优先权来被选择作为树支路。因此,树的定义在迭代中会改变。这个新方法保证了混合方程会有一个好的条件;然而,在每一次迭代中选择树造成了相当可观的额外负担,特别是如果使用了稀疏矩阵的方法。

一旦网络树被选择好,所有的混合分析方法会用近似一样的方式推导出基本的割集和环路方程系统。

\section{修正表格分析}
修正表格分析\cite{ref-28,ref-29}基于支路电压和支路电流形成电路方程。树是通过强迫所有的独立电压源成为树支路和所有的独立电流源成为树连接被选择的;剩余的树支路的选择会以一种被设计为使得F矩阵中的非零系数数目最少的方式进行\cite{ref-29}。电路方程随后从基本的割集方程,基本的环路方程,和支路关系推导得到,产生下面的方程系统
\begin{equation}
    \begin{bmatrix}
    I & 0 & 0 & F \\
    0 & I & -F^T & 0 \\
    C_{TIT} & C_{TOL} & C_{TUT} & C_{TIL} \\
    C_{LIT} & C_{LVL} & C_{LVT} & C_{LIL}
    \end{bmatrix}\begin{bmatrix}
    I_T \\
    V_L \\
    V_T \\
    I_L
    \end{bmatrix}=\begin{bmatrix}
    0 \\
    0 \\
    E \\
    J
    \end{bmatrix}
    \label{eq:3.32}
\end{equation}
八个C矩阵对于表示支路关系是必要的。因为(\ref{eq:3.32})中的单位阵,混合方程的约减集合可以从(\ref{eq:3.32})中获得如下
\begin{equation}
    \begin{bmatrix}
    C_{TUT} + C_{TUL}F^T & C_{TIL}-C_{TIT}F \\
    C_{LUT} + C_{LUL}F^T & C_{LIL}-C_{LIT}F
    \end{bmatrix}\begin{bmatrix}
    V_T \\
    I_L
    \end{bmatrix}=\begin{bmatrix}
    E \\
    J
    \end{bmatrix}
    \label{eq:3.33}
\end{equation}
(\ref{eq:3.33})中指示的矩阵操作是无关紧要的,因为F和$-F^t$是系数要不为0要不为$\pm 1$的整数矩阵。

\begin{figure}[htbp]
\small
    \centering
    \includegraphics[width=0.7\textwidth]{figure/Chapter3/图3.5.png}
    \caption{示例电路的修正表格。}
    \label{图3.5}
\end{figure}

作为MTA公式的一个例子,图\ref{图3.3}所示电路的完整的修正表格如图\ref{图3.5}所示;选中的树包含$V_1$,$C_4$,和$C_5$。这个例子电路方程约减的集合为
\begin{equation}
    \begin{bmatrix}
    1 & 0 & 0 & 0 & 0 & 0 & 0 \\
    0 & C_4\frac{d}{dt} & 0 & -1 & 1 & 1 & 1 \\
    0 & 0 & C_5\frac{d}{dt} & 0 & 0 & -1 & -1 \\
    -1 & 1 & 0 & R_2 & 0 & 0 & 0 \\
    0 & -1 & 0 & 0 & R_3 & 0 & 0 \\
    0 & -g_6 & 0 & 0 & 0 & 1 & 0 \\
    0 & 1 & -1 & 0 & 0 & 0 & R_7
    \end{bmatrix}\begin{bmatrix}
    V_1 \\
    V_4 \\
    V_5 \\
    I_2 \\
    I_3 \\
    I_6 \\
    I_7
    \end{bmatrix}=\begin{bmatrix}
    E_1 \\
    0 \\
    0 \\
    0 \\
    0 \\
    0 \\
    0
    \end{bmatrix}
    \label{eq:3.34}
\end{equation}
方程(\ref{eq:3.34})指出了MTA公式的一个严重缺陷。在DC分析中,(\ref{eq:3.34})中的时间导数项是0,所以因此第二个和第三个对角项是0。为了求得DC解,选主元必须被用来获得非零对角项。

MTA公式被用在DIFFPAIR,RTLINV,和TTLINV电路中,这些电路是用来测试节点分析法的那十个基准电路中的一部分。这三种电路的结果与MNA结果的对比在表\ref{表3.4}中列出。对于这三种简单的电路,MTA公式产生的方程系统含有两倍的非零矩阵项的数目;而且,求解MTA方程要求的计算代价是求解MNA方程要求的工作量的三倍。此外,减少原本$2b \times 2b$修正表格要求的计算负担并没有在表\ref{表3.4}中加入作对比。在Ho等人\cite{ref-47}的文章中,MNA公式和MTA公式的一些对比表明MTA方程中的非零系数的数目是MNA\footnote{原文中误写成了MHA}方程中的非零系数数目的七倍。然而,计算原本修正表格中的系数的数目是不公正的,因为整数系数没有被存储\cite{ref-29};因此,更有意义的对比是如表\ref{表3.4}中给出的约减后 的方程的对比。
\begin{figure}[htbp]
\small
    \centering
    \includegraphics[width=0.7\textwidth]{figure/Chapter3/表3.4.png}
    \caption{MTA公式和MNA公式的对比。}
    \label{表3.4}
\end{figure}

这里提到的数据一点也不好理解;然而,这个数据确实表明专用的MTA公式和通用的混合分析方法,在每行非零系数方面,产生了与节点分析方程稀疏度可比拟的方程系统。然而,混合分析方法产生了一个范围更大的方程集合,所以因此混合方程的非零系数总数会比节点方程的大。树选择的额外复杂度在计算代价方面不会产生明显的节省;极其相反的是,混合方程倒是真正要求更多的计算代价来求解。

\section{稀疏表格建立}
在本章中考虑的第三种构建技术是稀疏表格公式的通用技术\cite{ref-48}。通过这种方法,所有电路变量被包含进一个未知向量,而且会组建一个非常大的,非常稀疏的电路方程系统。不需要对该方程集合进行之前的约减或者简化;相反,求解的是表格方程的整个系统。

稀疏表格未知向量包含了节点电压,支路电压,支路电流,电容电荷和电感磁通量。方程由KCL和KVL,以及支路关系决定,如下。
\begin{equation}
    (KCL) AI_b = 0
    \label{eq:3.35}
\end{equation}

\begin{equation}
    (KVL) A^TV_n = V_b
    \label{eq:3.36}
\end{equation}

和之前一样,Kirchoff定律构成了b个约束;其余的方程由支路关系提供。产生的方程集合如图\ref{图3.6}所示。C矩阵再次成为系数矩阵,由支路关系决定。
\begin{figure}[htbp]
\small
    \centering
    \includegraphics[width=0.7\textwidth]{figure/Chapter3/图3.6.png}
    \caption{稀疏表格。}
    \label{图3.6}
\end{figure}

无论是节点分析公式还是混合分析都可以从表格的适当简化推导得到\cite{ref-46}。然而,稀疏表格公式的基本概念是获得更紧凑的方程集合不需要约减或者消去。相反,求解完整的方程系统。很明显,与确定节点方程或者混合方程的解对比,通过传统方法获得表格方程的解要求巨量的计算代价。因此,在方程求解算法中附加一些精明的处理办法是必要的。

表格矩阵的所有系数都属于下面四种变量类型中的一种:
\begin{enumerate}
    \item 拓扑性的 +1或者-1
    \item 常量的 独立的时间或者未知向量
    \item 时间相关的 依赖于时间的
    \item 非线性的 依赖于未知向量的
\end{enumerate}
例如,线性电阻是常量的;线性电容的等效电导是时间相关的;二极管或者非线性电容的等效电导是非线性的。为了在每步Newton迭代中确定解,只有包含了非线性系数的计算需要被执行,因为只有非线性系数会从一次Newton迭代到下一次变化。包含时间相关系数的操作在每个时间点上只需要被执行一次。最后,包含拓扑系数和常量系数的操作只需要被执行一次。

表格中行和列被消去的顺序由一种被设计为最小化总计算代价的方式决定。包含非线性系数的操作会得到比时间相关操作更高的权重,而时间相关操作反过来被给予了比常量的或者拓扑的操作更高的权重。这样的一种重排序算法很明显比节点分析或者混合分析要求的方法更复杂。

稀疏表格公式的基本缺陷是病态条件的内在问题。在重排序这一步,没有线索表明哪个系数作为主元是数值稳定的。即使重排序策略通过部分选主元技术被修正得试着决定表格方程,该选主元策略必须依赖“平均”值。因为电路中的很多非线性的等效电导的变化范围是$10^3$到$10^{-12}$ mhos,所以平均电导的概念明显将带来问题。在遇到的病态条件问题的事件中,整个重排序算法在建立一个更好条件的消去顺序时必须被唤醒。

稀疏表格技术不能被期待总可以产生出一套数值条件好的方程。为了抵消表格的规模,复杂的重排序和求解算法是必要的。最后,稀疏表格求解要求的操作数量被报道只比节点分析公式的\cite{ref-46}少10\%-20\%。计算代价上这少量的一点提升很难看起来与该技术的巨量复杂性以及遇到的病态条件的前景相称。

\section{总结}
电路仿真要求,作为第一步,通过电路方程系统表征物理电路。这些方程必须满足支路关系的约束和Kirchoff拓扑定律。方程构建算法决定了满足这两个约束的电路方程的独立系统。

方程构建算法绝不是唯一的。构建算法中的差别基本依赖于被用来构建方程系统的未知量的向量。因为任何产生独立方程系统的未知向量都可以被使用,所以有无数不同的方程构建技术。两种基本的方法是割集分析,基于支路电压的未知向量构建方程,和环路分析,基于支路电流的未知向量构建方程。因为这两种基本的方程构建方法对网络中的支路类型施加了限制,所以没有一种方法适用于实际的仿真问题。相反,实际的构造方法是割集分析和环路分析的结合。

三种目前受欢迎的构造方法是节点分析,混合分析,和稀疏表格分析。所有三种算法在兼容大体上所有的电路配置方面都有充分的通用性。特定算法的选择由求解结果电路方程系统要求的计算代价,以及电路方程系统的数值条件决定。提到的节点分析方法可以产生一种好条件的方程系统,只要最少量的计算代价就可以被求解。混合分析方法,作为对比,在DC分析的例子上遭遇明显的数值条件问题。而且,与MNA方法比较,提到的混合分析方法产生的方程系统要求大量的计算代价求解。最后,提到的稀疏表格方法,尽管整体看起来通用,但是要求更复杂的编程,而且,会产生出病态条件的方程系统。目前,稀疏表格方法在计算代价方面只能得到10\%到20\%的减少。

出于这些原因,节点分析在两个SPICE版本中都被作为了构建算法。SPICE的第一个版本包含了节点分析的CANCER修正,当电路包含少量浮动电压源和/或电感时,该法工作地很好。MNA1构建方法目前在SPICE2中实现了。尽管MNA2方法更有效,但是行交换带来的额外的复杂度以及非对称的系数矩阵使得MNA2方法缺少了吸引力。而且,MNA1算法的指针系统是MNA2方法要求的指针系统规模的一半。
